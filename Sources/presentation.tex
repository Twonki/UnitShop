\documentclass[12pt]{beamer}
\usepackage[utf8]{inputenc}
\usepackage[T1]{fontenc}
\usepackage{lmodern}
\usepackage[english]{babel}
\usepackage{listings}
\usetheme{default}

\begin{document}
	\author{Leonhard Applis}
	\title{Unit-Testing Essentials}
	\begin{frame}[plain]
		\maketitle
	\end{frame}
	
	\section{Introduction}
	
	\begin{frame}
		\frametitle{What is a Unit-Test?}
		\framesubtitle{Definition}
		A Unit Test ... 
		\begin{enumerate}
			\item tests a single piece of functionality
			\item is written by the developer
			\item should not depend on any other, external resource
			\item should be run with every change of the code
		\end{enumerate}
	\end{frame}

	\begin{frame}
		\frametitle{Unit-Test Examples}
		These are Unit Tests ...
		\begin{itemize}
			\item testing the distance-function of a point to another point
			\item testing whether the sort() function of your list sorts the list 
			\item testing whether a constructor fails for null-values
		\end{itemize}
		~\newline
		These are \textbf{not} Unit Tests...
		\begin{itemize}
			\item read and write to an database 
			\item sending an http-request to an api 
			\item checking if an config-file is successfully read
		\end{itemize}
	\end{frame}

	\begin{frame}
		\frametitle{XUnit}
		\begin{itemize}
			\item A common way of writing unit tests
			\item NUnit, JUnit, HUnit, ...
			\item tests are written as functions, which have an assert statement
			\item if the assert succeeds, the test is green
			\item if the assert fails, the test is red and the difference is shown
			\item as functions, tests can be debugged
		\end{itemize}
	\end{frame}
	
	\begin{frame}[fragile]
		\frametitle{Example 1: Point Distance Test}
		\begin{scriptsize}
			\lstinputlisting[language=Java]{./CodeSnippets/DistanceTests.java}	
		\end{scriptsize}
	\end{frame}
	
	\begin{frame}[fragile]
		\frametitle{Example 1: Point Distance Code}
		\begin{small}
			\lstinputlisting[language=Java]{./CodeSnippets/Point.java}	
		\end{small}
	\end{frame}
	
	\begin{frame}
		\frametitle{JUnit markup words}
		\begin{itemize}
			\item \textit{@Test} marks that the following function is a Test
			\item \textit{assertEquals(expected,actual)} is used to check values - warning: for objects the \textit{equals()} method is used 
			\item \textit{assertFalse(...),assertTrue(...),assertThrowsException(...)} are shortcuts to write smaller and more expressive tests
			\item \textit{@Tag "xy"} gives the test a filterable tag, e.g. you could run only \textit{player-related} tests
		\end{itemize}
	\end{frame}

	\section{Best Practices}

	\begin{frame}
		\frametitle{Best Practice - No Global Dependencies}
		Do not use (static) global variables in your tests if it's not absolutely necessary! 
		
		Reasons: 
		\begin{itemize}
			\item Tests may fail because of their order
			\item if the dependency changes, every tests needs to change
			\item functions like \textit{resetDB()} after every tests maybe fail because of parallel execution
			\item deactivating parallel-execution is bad
			\item the items you need are not written in your tests, but have to be looked up
		\end{itemize}
	
		Mitigation: \textbf{create the dependency in every test}. If you have a common pattern, you may want to use a \textit{factory method} which returns a new, default instance of your  dependency.
	\end{frame}
	
	\begin{frame}
		\frametitle{One 'Test' per Test}
		Only do \textit{one Assert per test!} \newline
		If you are doing multiple asserts, you usually also have multiple tests. 
		
		If you try to assert two things in one statement, you are not sure if the later one would be right if the first one failed.
		~\newline
		\textbf{negative:} A lot more tests, A lot more code in general
		~\newline
		\textbf{positive:} Clearer tests, tests are more maintainable, ~\newline more test mean more green tests, which makes you happy. 
	\end{frame}
	
	\begin{frame}
		\frametitle{Arrange - Act - Assert}
		Write your tests with a common pattern: 
		\begin{enumerate}
			\item \textbf{Arrange:} Create every item required for the test
			\item \textbf{Act:} Perform every Action required for the test
			\item \textbf{Assert:} Perform the Check(s)
		\end{enumerate}
		~\newline
		Split those points by a single line, and you'll have a nice, common pattern. \newline
		Don't do \textit{Arrange-Act-Rearrange-Act-Assert!} ~\newline
		Never do \textit{Arrange-Act-Assert-Act-Assert!}
	\end{frame}
	
	\section{Stubs and Mocks}
	
	\begin{frame}
		\frametitle{Working with dependent Objects}
		For Inter-Object communication there are usually 2 cases:
		\begin{enumerate}
			\item To do something, I need another object
			\item My actions change another object
		\end{enumerate}
		No matter if these items are global (variables), static (System Args) or external (such as databases) 
	\end{frame}

	\begin{frame}
		\frametitle{Stubs}
		A stub is a testclass which implements a given interface, which can be used as \textit{fakeInput} for another test. 
		
		~\newline
		A stub is \textit{input only}, so the stub will not be changed after generation. \textbf{A stub provides behaviour}. 
		~\newline
		e.g: a function \textit{addPlayer(InterfacePlayer)} from the class \textit(team) may use a stubPlayer
		
		~\newline
		\textbf{primary reason for stubs:} stubs act the way you want, without having any logic. The tests of \textit{team} would maybe fail if the \textit{RealPlayer} would have failures. 
	\end{frame}

	\begin{frame}
		\frametitle{Example 2: StubPlayer}
		\begin{small}
			\lstinputlisting[language=Java]{./CodeSnippets/Player.java}
			\lstinputlisting[language=Java]{./CodeSnippets/StubPlayer.java}	
		\end{small}
	\end{frame}

	\begin{frame}
		\frametitle{Example 2: SimpleTeam}
		\begin{small}
			\lstinputlisting[language=Java]{./CodeSnippets/SimpleTeam.java}	
		\end{small}
	\end{frame}

	\begin{frame}
		\frametitle{Example 2: Team-Tests}
		\begin{scriptsize}
			\lstinputlisting[language=Java]{./CodeSnippets/SimpleTeamTest.java}	
		\end{scriptsize}
	\end{frame}

	\begin{frame}
		\frametitle{Stubs: Why Interfaces are your friends}
		To do a real nice set of stubs, you need interfaces. e.g. \textit{Player}. 
		~\newline
		Interfaces \textit{clean your code}, analyze and generalize behavior. You can tell what a class should be doing by the implemented interfaces (atleast with good names used).
		~\newline
		In object oriented languages (such as C-Sharp and Java) it is considered best practice to only use interfaces as inputs (especially in interfaces) which is called \textit{programming against interfaces}.
	\end{frame} 

	\begin{frame}
		\frametitle{Mocks}
		A mock is somewhat the opposite of a stub: 
		The stub provides behavior, the mock \textit{takes} behavior.
		
		~\newline
		A mock is considered \textit{write only}, except for the asserts. 
		
		~\newline
		Example: The function \textit{addToTeam(Team)} of the class \textit{Player} would need a \textit{MockTeam}. 
		The mock team \textit{safes} that the class was touched in the expected way.
		
		~\newline
		\textbf{warning:} in some languages, such as JS, everything is called a mock. 
	\end{frame}

	\begin{frame}
		\frametitle{Example 3: MockTeam}
		\begin{small}
			\lstinputlisting[language=Java]{./CodeSnippets/Team.java}
			\lstinputlisting[language=Java]{./CodeSnippets/MockTeam.java}	
		\end{small}
	\end{frame}
	
	\begin{frame}
		\frametitle{Example 3: SimplePlayer}
		\begin{small}
			\lstinputlisting[language=Java]{./CodeSnippets/SimplePlayer.java}	
		\end{small}
	\end{frame}
	
	\begin{frame}
		\frametitle{Example 3: Team-Tests}
		\begin{scriptsize}
		\lstinputlisting[language=Java]{./CodeSnippets/SimplePlayerTest.java}	
		\end{scriptsize}
	\end{frame}

	\begin{frame}
		\frametitle{Stubs vs. Mocks vs. Fakes}
		
		\begin{itemize}
			\item \textbf{A Stub} provides behavior and is \textit{read only}. 
			\item \textbf{A Mock} accepts behavior and actions and is \textbf{write only}
			\item \textbf{A Fake} is a combination of a stub and a mock. 
		\end{itemize}
	
		~\newline
		You may consider calling everything a fake, as the term \textit{Mock} is overloaded (and misused) in some languages and frameworks, while the term stub is unknown to a lot of people.  
	\end{frame}
	
	\section{TestStrategy: Test-First vs Test-Last-If-Ever}
	
	\begin{frame}
		\frametitle{Naming - Good vs. Bad}
		Names of the tests are very important for readability of reports. 
		example bad-names: \textit{testCalculate, testCalculateFails, testCollide, testCollide2 ...}
		
		the name of the test should include:
		\begin{enumerate}
			\item the function under test
			\item a summary of the setup 
			\item the expected result
		\end{enumerate}
	
		simple test-name-pattern for beginners: \newline
		\textit{testFunction-WithBadInput-shouldThrowError} \newline
		\textit{testConstructor-withValidAttributes-shouldBeBuild}
	\end{frame}
	
	\begin{frame}
		\frametitle{Test First}
		The primary idea of \textit{test first} is to write unit-tests for the expected behavior without writing any line of the \textit{real code}. 
		~\newline
		Normal workflow: \textit{write code -> extract interface -> write tests}. 
		~\newline
		Test First worklow: \textit{write interface -> write tests -> write code}

		~\newline
		\textbf{benefits:} The requirements and behavior must be defined beforehand, the code is properly tested, the code is using interfaces.
		~\newline
		\textbf{downsides:} If the class is complex, the design and testing may take very long. 
		
		The tests may be red for a long time if the code is complex. 	

	\end{frame}

	\begin{frame}
		\frametitle{Test-Driven-Development}
		\framesubtitle{RED-GREEN-REFACTOR}
		\textbf{TDD} is a specified subclass of \textit{test-first}. 
		\begin{enumerate}
			\item pick an item from the todo list
			\item write a non-passing test
			\item write code to pass the test
			\item clean up the class until it looks nice, but keep the tests as-is
			\item repeat
		\end{enumerate}
		\textbf{Where you should never have more than one red test!}
	
	
		~\newline
		downsides: refactoring is sometimes left out, not appliable for complex problems - only for \textit{commodity code}.
	\end{frame}

	\begin{frame}
		\frametitle{TDD-Schools}
		\begin{enumerate}
			\item \textbf{London-School:} \textit{Mock Everything} \newline Mocking forces nice interfaces and code that uses nice interfaces.  
			\item \textbf{Chicago-School:} \textit{Mock Nothing} 
			\newline Mocking bloats the code, and makes sometimes no sense. 
			\item \textbf{Munich-School:}\textit{Mock what makes sense} 
			\newline dont focus on a school - do what goes best and respect the language 
		\end{enumerate}
	\end{frame}
	
\end{document}