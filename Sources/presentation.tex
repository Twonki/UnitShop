\documentclass[12pt]{beamer}
\usepackage[utf8]{inputenc}
\usepackage[T1]{fontenc}
\usepackage{lmodern}
\usepackage[english]{babel}
\usetheme{default}

\begin{document}
	\author{Leonhard Applis}
	\title{Unit-Testing Essentials}
	%\subtitle{}
	%\logo{}
	%\institute{}
	%\date{}
	%\subject{}
	%\setbeamercovered{transparent}
	%\setbeamertemplate{navigation symbols}{}
	\begin{frame}[plain]
		\maketitle
	\end{frame}
	
	\section{Introduction}
	
	\begin{frame}
		\frametitle{What is a Unit-Test?}
		\framesubtitle{Definition}
		A Unit Test ... 
		\begin{enumerate}
			\item tests a single piece of functionality
			\item is written by the developer
			\item should not depend on any other, external resource
			\item should be run with every change of the code
		\end{enumerate}
	\end{frame}

	\begin{frame}
		\frametitle{Unit-Test Examples}
		These are Unit Tests ...
		\begin{itemize}
			\item testing the distance-function of a point to another point
			\item testing whether the sort() function of your list sorts the list 
			\item testing whether a constructor fails for null-values
		\end{itemize}
		These are \textbf{not} Unit Tests...
		\begin{itemize}
			\item read and write to an database 
			\item sending an http-request to an api 
			\item checking if an config-file is successfully read
		\end{itemize}
	\end{frame}

	\begin{frame}
		\frametitle{XUnit}
		\begin{itemize}
			\item A common way of writing unit tests
			\item NUnit, JUnit, HUnit, ...
			\item tests are written as functions, which have an assert statement
			\item if the assert succeeds, the test is green
			\item if the assert fails, the test is red and the difference is shown
			\item as functions, tests can be debugged
		\end{itemize}
	\end{frame}
	
	\begin{frame}
		\frametitle{Example 1: Point Distance Test}
	\end{frame}
	
	\begin{frame}
		\frametitle{Example 1: Point Distance Code}
	\end{frame}
	
	\begin{frame}
		\frametitle{JUnit markup words}
		\begin{itemize}
			\item \textit{@Test} marks that the following function is a Test
			\item \textit{assertEquals(expected,actual)} is used to check values - warning: for objects the \textit{equals()} method is used 
			\item \textit{assertFalse(...),assertTrue(...),assertThrowsException(...)} are shortcuts to write smaller and more expressive tests
			\item \textit{@Tag "xy"} gives the test a filterable tag, e.g. you could run only \textit{fast} tests
		\end{itemize}
	\end{frame}

	\section{Best Practices}

	\begin{frame}
		\frametitle{Best Practice - No Global Dependencies}
	\end{frame}
	
	\begin{frame}
		\frametitle{One 'Test' per Test}
	\end{frame}
	
	\begin{frame}
		\frametitle{Arrange - Act - Assert}
	\end{frame}
	
	\section{Stubs and Mocks}
	
	\begin{frame}
		\frametitle{Working with dependent Objects}
		For Inter-Object communication there are usually 2 cases:
		\begin{enumerate}
			\item To do something, I need another object
			\item My actions change another object
		\end{enumerate}
		No matter if these items are global (variables), static (System Args) or external (such as databases) 
	\end{frame}

	\begin{frame}
		\frametitle{Stubs}
	\end{frame}

	\begin{frame}
		\frametitle{Example 2: UpdateManager}
	\end{frame}
	
	\begin{frame}
		\frametitle{Stubs: Why Interfaces are your friends}
	\end{frame}

	\begin{frame}
		\frametitle{Mocks}
	\end{frame}

	\begin{frame}
		\frametitle{Example 3}
	\end{frame}
	
	\begin{frame}
		\frametitle{Stubs vs. Mocks vs. Fakes}
	\end{frame}
	
	\section{TestStrategy: Test-First vs Test-Last-If-Ever}
	
	\begin{frame}
		\frametitle{Naming - Good vs. Bad}
	\end{frame}
	
	\begin{frame}
		\frametitle{Test First}
	\end{frame}

	\begin{frame}
		\frametitle{Test-Driven-Development}
		\framesubtitle{RED-GREEN-REFACTOR}
	\end{frame}

	\begin{frame}
		\frametitle{TDD-Schools}
		\begin{enumerate}
			\item \textbf{London-School:} "Mock Everything" \newline Mocking forces nice interfaces and code that uses nice interfaces. 
			\item \textbf{Chicago-School:} "Mock Nothing" 
			\newline Mocking bloats the code, and makes sometimes no sense. 
			\item \textbf{Munich-School:}" "Mock what makes sense" 
			\newline don´t focus on a "school" - do what goes best and respect the language 
		\end{enumerate}
	\end{frame}
	
\end{document}